%%%%%%%%%%%%%%%%%%%%%%%%%%%%%%%%%%%%%%%%%%%%%%%%%%%%%%%%%%%%%%%%%%%%%%%%%%%%%%%%
% usepackage
%%%%%%%%%%%%%%%%%%%%%%%%%%%%%%%%%%%%%%%%%%%%%%%%%%%%%%%%%%%%%%%%%%%%%%%%%%%%%%%%
\usepackage{amsmath, amssymb}
\usepackage{mathrsfs}
  % \mathscr
\usepackage{ascmac}
  % \smallskip
\usepackage{braket}
  % \set, \Set
\usepackage{physics}
  % \pdv
  % \bra, \ket
\usepackage{mathtools}
  % \xleftrightarrow
\usepackage{graphics}
  % \rotatebox
\usepackage{bm}
\usepackage{bbold}
\usepackage{color}
\usepackage{otf}
  % ajRoman
\usepackage{here}
  % 大文字のHを使用することで好きな位置に図を配置
  % \begin{figure}[H]
% \usepackage{xstring}

  % \IfSubStr
%%%


%%%%%%%%%%%%%%%%%%%%%%%%%%%%%%%%%%%%%%%%%%%%%%%%%%%%%%%%%%%%%%%%%%%%%%%%%%%%%%%%
% newtheorem
%%%%%%%%%%%%%%%%%%%%%%%%%%%%%%%%%%%%%%%%%%%%%%%%%%%%%%%%%%%%%%%%%%%%%%%%%%%%%%%%
\usepackage{amsthm}
\theoremstyle{definition}

\newtheorem{Axiom}{公理}[section] % axiom
\newtheorem{Conj}{予想}[section] % conjecture
\newtheorem{Cor}{系}[section] % corollary
\newtheorem{Def}{定義}[section] % definition
\newtheorem{Ex}{例}[section] % example, exercise
\newtheorem{Lem}{補題}[section] % lemma
\newtheorem{Obs}{観察}[section] % observation
\newtheorem{pf}{証明}[section] % proof
\newtheorem{Prop}{命題}[section] % proposition
\newtheorem{Rmk}{注意}[section] % remark
\newtheorem{Thm}{定理}[section] % theorem



%%%%%%%%%%%%%%%%%%%%%%%%%%%%%%%%%%%%%%%%%%%%%%%%%%%%%%%%%%%%%%%%%%%%%%%%%%%%%%%%
% extended latex
%%%%%%%%%%%%%%%%%%%%%%%%%%%%%%%%%%%%%%%%%%%%%%%%%%%%%%%%%%%%%%%%%%%%%%%%%%%%%%%%
%%% skim
\newcommand{\smallskim}{\vspace{-1zh}}
\newcommand{\medskim}{\vspace{-2zh}}
\newcommand{\bigskim}{\vspace{-4zh}}

%%% insert image from web
\newcommand{\img}[2]{
  \write18{wget "#1" -O #2}
  \begin{figure}[h]
    \centering
    {\includegraphics[width=0.5\linewidth]{#2}}
    \href{#1}{src}
    \label{#2}
  \end{figure}
}



%%%%%%%%%%%%%%%%%%%%%%%%%%%%%%%%%%%%%%%%%%%%%%%%%%%%%%%%%%%%%%%%%%%%%%%%%%%%%%%%
% extended package{physics}
%%%%%%%%%%%%%%%%%%%%%%%%%%%%%%%%%%%%%%%%%%%%%%%%%%%%%%%%%%%%%%%%%%%%%%%%%%%%%%%%
\DeclareMathOperator{\Det}{Det}


%%% vert
\newcommand{\mv}{\,\middle\vert\,}
  % \qty(\frac{1}{d} \mv e) % for example


%%% generic braket <x, y> powered by usepackage{physics}
\DeclareDocumentCommand\genericBraket{ s o o o m g }
{
  % #1:star
  % #2:option - left bra
  % #3:option - middle sep
  % #4:option - right ket
  % #5:mandatory
  % #6:given inside
  \newcommand{\sss}{#1}
  \newcommand{\gbra}{\left#2} % generic bra
  \newcommand{\gsep}{\,#3\,} % generic sep
  \newcommand{\gket}{\right#4} % generic ket
  \newcommand{\mm}{#5}
  \newcommand{\nn}{#6}
	\IfBooleanTF{\sss}
	{ % No resize
		\IfNoValueTF{\nn}
		{\vphantom{\mm}\gbra\smash{\mm}\gsep\smash{\mm}\gket}
		{\vphantom{\mm\nn}\gbra\smash{\mm}\gsep\smash{\nn}\gket}
	}
	{ % Auto resize
		\IfNoValueTF{\nn}
		{\gbra{\mm}\gsep{\mm}\gket}
		{\gbra{\mm}\gsep{\nn}\gket}
	}
}

%%% <x,y>
\newcommand{\cinnerproduct}{\genericBraket[\lparen][,][\rparen]}
\DeclareDocumentCommand\cbraket{}{\cinnerproduct} % Alternative for \cinnerproduct
\DeclareDocumentCommand\cip{}{\cinnerproduct} % Shorthand for \cinnerproduct
%%% Hilbert-Schmidt inner product <x,y>_HS
\newcommand{\hsip}[2]{\cip{#1}{#2}_{\mathrm{HS}}}
%%% (x|y)
\newcommand{\pbraket}{\genericBraket[\lparen][\mv][\rparen]}
%%% (x||y)
\newcommand{\infodivBraket}{\genericBraket[\lparen][\middle\|][\rparen]}
%%% D(x||y)
\newcommand{\infodiv}[2]{D\infodivBraket{#1}{#2}}
%%% D_KL(x||y)
\newcommand{\kldiv}[2]{D_\mathrm{KL}\infodivBraket{#1}{#2}}


%%%%%%%%%%%%%%%%%%%%%%%%%%%%%%%%%%%%%%%%%%%%%%%%%%%%%%%%%%%%%%%%%%%%%%%%%%%%%%%%
% math
%%%%%%%%%%%%%%%%%%%%%%%%%%%%%%%%%%%%%%%%%%%%%%%%%%%%%%%%%%%%%%%%%%%%%%%%%%%%%%%%
%%% 体
\newcommand{\N}{\mathbb{N}}%自然数
\newcommand{\Z}{\mathbb{Z}}%整数
\newcommand{\Q}{\mathbb{Q}}%有理数
\newcommand{\R}{\mathbb{R}}%実数
\newcommand{\C}{\mathbb{C}}%複素数

% 下向き\in
\newcommand{\indown}{\rotatebox{90}{$\in$}}

% imaginary unit
\newcommand{\im}{\mathrm{i}}
\DeclareMathOperator*{\spn}{span}

% linear algebra
\newcommand{\trans}{^t{}}
\newcommand{\inv}{^{-1}{}}

\newcommand{\alart}[1]{\textcolor{red}{\footnotesize {#1}}}

\newcommand{\what}{\widehat}
\newcommand{\wtilde}{\wtilde}
\newcommand{\wbar}{\overline}
\newcommand{\ctrans}{^\dagger{}}
\newcommand{\expectation}{\mathrm{E}\qty}
\newcommand{\variance}{\mathrm{Var}\qty}
\newcommand{\qst}{\qq{s.t.}}
\newcommand{\qie}{\qq{i.e.}}
\newcommand{\qeg}{\qq{e.g.}}
\newcommand{\qwhere}{\qq{where}}
\newcommand{\vectorize}{\mathrm{vec}\qty}

%%% infomation geometry
\newcommand{\mrep}{^{(m)}{}}
\newcommand{\erep}{^{(e)}{}}


%%%%%%%%%%%%%%%%%%%%%%%%%%%%%%%%%%%%%%%%%%%%%%%%%%%%%%%%%%%%%%%%%%%%%%%%%%%%%%%%
% highlight
%%%%%%%%%%%%%%%%%%%%%%%%%%%%%%%%%%%%%%%%%%%%%%%%%%%%%%%%%%%%%%%%%%%%%%%%%%%%%%%%
\newcommand{\highlight}[2][yellow]{\tikz[baseline=(x.base)]{\node[rectangle,rounded corners,fill=#1!10](x){#2};}}
\newcommand{\highlightcap}[3][yellow]{\tikz[baseline=(x.base)]{\node[rectangle,rounded corners,fill=#1!10](x){#2} node[below of=x, color=#1]{#3};}}
