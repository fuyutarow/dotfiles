%%%%%%%%%%%%%%%%%%%%%%%%%%%%%%%%%%%%%%%%%%%%%%%%%%%%%%%%%%%%%%%%%%%%%%%%%%%%%%%%
% usepackage
%%%%%%%%%%%%%%%%%%%%%%%%%%%%%%%%%%%%%%%%%%%%%%%%%%%%%%%%%%%%%%%%%%%%%%%%%%%%%%%%
\usepackage{amsmath, amssymb}
\usepackage{mathrsfs}
  % \mathscr
\usepackage{ascmac}
  % \smallskip
\usepackage{braket}
  % \set, \Set
\usepackage{physics}
  % \pdv
  % \bra, \ket
\usepackage{mathtools}
  % \xleftrightarrow
\usepackage{graphics}
  % \rotatebox
\usepackage{bm}
\usepackage{bbold}
\usepackage{color}
\usepackage{otf}
  % ajRoman
\usepackage{here}
  % 大文字のHを使用することで好きな位置に図を配置
  % \begin{figure}[H]
% \usepackage{xstring}

  % \IfSubStr
%%%


%%%%%%%%%%%%%%%%%%%%%%%%%%%%%%%%%%%%%%%%%%%%%%%%%%%%%%%%%%%%%%%%%%%%%%%%%%%%%%%%
% newtheorem
%%%%%%%%%%%%%%%%%%%%%%%%%%%%%%%%%%%%%%%%%%%%%%%%%%%%%%%%%%%%%%%%%%%%%%%%%%%%%%%%
\usepackage{amsthm}
\theoremstyle{definition}

\newtheorem{Axiom}{公理}[section] % axiom
\newtheorem{Conj}{予想}[section] % conjecture
\newtheorem{Cor}{系}[section] % corollary
\newtheorem{Def}{定義}[section] % definition
\newtheorem{Ex}{例}[section] % example, exercise
\newtheorem{Lem}{補題}[section] % lemma
\newtheorem{Obs}{観察}[section] % observation
\newtheorem{pf}{証明}[section] % proof
\newtheorem{Prop}{命題}[section] % proposition
\newtheorem{Rmk}{注意}[section] % remark
\newtheorem{Thm}{定理}[section] % theorem



%%%%%%%%%%%%%%%%%%%%%%%%%%%%%%%%%%%%%%%%%%%%%%%%%%%%%%%%%%%%%%%%%%%%%%%%%%%%%%%%
% Timestamp on header and footer
%%%%%%%%%%%%%%%%%%%%%%%%%%%%%%%%%%%%%%%%%%%%%%%%%%%%%%%%%%%%%%%%%%%%%%%%%%%%%%%%
%   \usepackage{fancyhdr,datetime2}
%   \pagestyle{fancy}
%   \renewcommand{\headrulewidth}{0pt}
%   \rhead{\thepage}
%   \chead{}
%   \lhead{}
%   \cfoot{\texttt{\jobname, draft of \DTMnow}}%\today\ at \now}}



%%%%%%%%%%%%%%%%%%%%%%%%%%%%%%%%%%%%%%%%%%%%%%%%%%%%%%%%%%%%%%%%%%%%%%%%%%%%%%%%
% extended latex
%%%%%%%%%%%%%%%%%%%%%%%%%%%%%%%%%%%%%%%%%%%%%%%%%%%%%%%%%%%%%%%%%%%%%%%%%%%%%%%%
%%% skim
\newcommand{\smallskim}{\vspace{-1zh}}
\newcommand{\medskim}{\vspace{-2zh}}
\newcommand{\bigskim}{\vspace{-4zh}}

%%% insert image from web
\newcommand{\img}[2]{
  \write18{wget "#1" -O #2}
  \begin{figure}[h]
    \centering
    {\includegraphics[width=0.5\linewidth]{#2}}
    \href{#1}{src}
    \label{#2}
  \end{figure}
}



%%%%%%%%%%%%%%%%%%%%%%%%%%%%%%%%%%%%%%%%%%%%%%%%%%%%%%%%%%%%%%%%%%%%%%%%%%%%%%%%
% extended package{physics}
%%%%%%%%%%%%%%%%%%%%%%%%%%%%%%%%%%%%%%%%%%%%%%%%%%%%%%%%%%%%%%%%%%%%%%%%%%%%%%%%
\DeclareMathOperator{\Det}{Det}

%%% vert
\newcommand{\mv}{\,\middle\vert\,}
  % \qty(\frac{1}{d} \mv e) % for example


%%% comma separate inner product <x, y> powered by usepackage{physics}
  \newcommand{\cinnerproductDelimiter}{,}
  \DeclareDocumentCommand\cinnerproduct{ s m g }
  { % comma Inner product
  	\IfBooleanTF{#1}
  	{ % No resize
  		\IfNoValueTF{#3}
  		{\vphantom{#2}\left\langle\smash{#2}\cinnerproductDelimiter\smash{#2}\right\rangle}
  		{\vphantom{#2#3}\left\langle\smash{#2}\cinnerproductDelimiter\smash{#3}\right\rangle}
  	}
  	{ % Auto resize
  		\IfNoValueTF{#3}
  		{\left\langle{#2}\cinnerproductDelimiter{#2}\right\rangle}
  		{\left\langle{#2}\cinnerproductDelimiter{#3}\right\rangle}
  	}
  }
  \DeclareDocumentCommand\cbraket{}{\cinnerproduct} % Alternative for \cinnerproduct
  \DeclareDocumentCommand\cip{}{\cinnerproduct} % Shorthand for \cinnerproduct
  \newcommand{\hsip}[2]{\cip{#1}{#2}_{\mathrm{HS}}}
%%%



%%% comma separate inner product <x, y> powered by usepackage{physics}
  \DeclareDocumentCommand\pbraket{ s m g }
  { % comma Inner product
    \newcommand{\sep}{\,\middle\vert\,}
  	\IfBooleanTF{#1}
  	{ % No resize
  		\IfNoValueTF{#3}
  		{\vphantom{#2}\left\lparen\smash{#2}\pbraketDelimiter\smash{#2}\right\rparen}
  		{\vphantom{#2#3}\left\lparen\smash{#2}\pbraketDelimiter\smash{#3}\right\rparen}
  	}
  	{ % Auto resize
  		\IfNoValueTF{#3}
  		{\left\lparen{#2}\pbraketDelimiter{#2}\right\rparen}
  		{\left\lparen{#2}\pbraketDelimiter{#3}\right\rparen}
  	}
  }
%%%


%%% comma separate inner product <x, y> powered by usepackage{physics}
  \DeclareDocumentCommand\infodivparen{ s m g }
  { % comma Inner product
    \newcommand{\sep}{\,\middle\|\,}
  	\IfBooleanTF{#1}
  	{ % No resize
  		\IfNoValueTF{#3}
  		{\vphantom{#2}\left\lparen\smash{#2}\sep\smash{#2}\right\rparen}
  		{\vphantom{#2#3}\left\lparen\smash{#2}\sep\smash{#3}\right\rparen}
  	}
  	{ % Auto resize
  		\IfNoValueTF{#3}
  		{\left\lparen{#2}\sep{#2}\right\rparenp}
  		{\left\lparen{#2}\sep{#3}\right\rparen}
  	}
  }
  \newcommand{\infodiv}[2]{D\infodivparen{#1}{#2}}
  \newcommand{\kldiv}[2]{D_\mathrm{KL}\infodivparen{#1}{#2}}
%%%

%%%%%%%%%%%%%%%%%%%%%%%%%%%%%%%%%%%%%%%%%%%%%%%%%%%%%%%%%%%%%%%%%%%%%%%%%%%%%%%%
% math
%%%%%%%%%%%%%%%%%%%%%%%%%%%%%%%%%%%%%%%%%%%%%%%%%%%%%%%%%%%%%%%%%%%%%%%%%%%%%%%%
%%% 体
\newcommand{\N}{\mathbb{N}}%自然数
\newcommand{\Z}{\mathbb{Z}}%整数
\newcommand{\Q}{\mathbb{Q}}%有理数
\newcommand{\R}{\mathbb{R}}%実数
\newcommand{\C}{\mathbb{C}}%複素数

% 下向き\in
\newcommand{\indown}{\rotatebox{90}{$\in$}}

% imaginary unit
\newcommand{\im}{\mathrm{i}}
\DeclareMathOperator*{\spn}{span}

% linear algebra
\newcommand{\trans}{^t{}}
\newcommand{\inv}{^{-1}{}}

\newcommand{\alart}[1]{\textcolor{red}{\footnotesize {#1}}}

\newcommand{\what}{\widehat}
\newcommand{\wtilde}{\wtilde}
\newcommand{\wbar}{\overline}
\newcommand{\ctrans}{^\dagger{}}
\newcommand{\expectation}{\mathrm{E}\qty}
\newcommand{\variance}{\mathrm{Var}\qty}
\newcommand{\qst}{\qq{s.t.}}
\newcommand{\qie}{\qq{i.e.}}
\newcommand{\qeg}{\qq{e.g.}}
\newcommand{\qwhere}{\qq{where}}
\newcommand{\vectorize}{\mathrm{vec}\qty}

%%% infomation geometry
\newcommand{\mrep}{^{(m)}{}}
\newcommand{\erep}{^{(e)}{}}

\DeclarePairedDelimiterX{\infdivx}[2]{(}{)}{%
  #1\;\delimsize\|\;#2%
}
\newcommand{\infdiv}{D\infdivx}


\newcommand{\highlight}[2][yellow]{\tikz[baseline=(x.base)]{\node[rectangle,rounded corners,fill=#1!10](x){#2};}}
\newcommand{\highlightcap}[3][yellow]{\tikz[baseline=(x.base)]{\node[rectangle,rounded corners,fill=#1!10](x){#2} node[below of=x, color=#1]{#3};}}


